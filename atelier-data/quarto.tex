% Options for packages loaded elsewhere
\PassOptionsToPackage{unicode}{hyperref}
\PassOptionsToPackage{hyphens}{url}
\PassOptionsToPackage{dvipsnames,svgnames,x11names}{xcolor}
%
\documentclass[
  letterpaper,
  DIV=11,
  numbers=noendperiod]{scrartcl}

\usepackage{amsmath,amssymb}
\usepackage{iftex}
\ifPDFTeX
  \usepackage[T1]{fontenc}
  \usepackage[utf8]{inputenc}
  \usepackage{textcomp} % provide euro and other symbols
\else % if luatex or xetex
  \usepackage{unicode-math}
  \defaultfontfeatures{Scale=MatchLowercase}
  \defaultfontfeatures[\rmfamily]{Ligatures=TeX,Scale=1}
\fi
\usepackage{lmodern}
\ifPDFTeX\else  
    % xetex/luatex font selection
\fi
% Use upquote if available, for straight quotes in verbatim environments
\IfFileExists{upquote.sty}{\usepackage{upquote}}{}
\IfFileExists{microtype.sty}{% use microtype if available
  \usepackage[]{microtype}
  \UseMicrotypeSet[protrusion]{basicmath} % disable protrusion for tt fonts
}{}
\makeatletter
\@ifundefined{KOMAClassName}{% if non-KOMA class
  \IfFileExists{parskip.sty}{%
    \usepackage{parskip}
  }{% else
    \setlength{\parindent}{0pt}
    \setlength{\parskip}{6pt plus 2pt minus 1pt}}
}{% if KOMA class
  \KOMAoptions{parskip=half}}
\makeatother
\usepackage{xcolor}
\setlength{\emergencystretch}{3em} % prevent overfull lines
\setcounter{secnumdepth}{5}
% Make \paragraph and \subparagraph free-standing
\makeatletter
\ifx\paragraph\undefined\else
  \let\oldparagraph\paragraph
  \renewcommand{\paragraph}{
    \@ifstar
      \xxxParagraphStar
      \xxxParagraphNoStar
  }
  \newcommand{\xxxParagraphStar}[1]{\oldparagraph*{#1}\mbox{}}
  \newcommand{\xxxParagraphNoStar}[1]{\oldparagraph{#1}\mbox{}}
\fi
\ifx\subparagraph\undefined\else
  \let\oldsubparagraph\subparagraph
  \renewcommand{\subparagraph}{
    \@ifstar
      \xxxSubParagraphStar
      \xxxSubParagraphNoStar
  }
  \newcommand{\xxxSubParagraphStar}[1]{\oldsubparagraph*{#1}\mbox{}}
  \newcommand{\xxxSubParagraphNoStar}[1]{\oldsubparagraph{#1}\mbox{}}
\fi
\makeatother

\usepackage{color}
\usepackage{fancyvrb}
\newcommand{\VerbBar}{|}
\newcommand{\VERB}{\Verb[commandchars=\\\{\}]}
\DefineVerbatimEnvironment{Highlighting}{Verbatim}{commandchars=\\\{\}}
% Add ',fontsize=\small' for more characters per line
\usepackage{framed}
\definecolor{shadecolor}{RGB}{241,243,245}
\newenvironment{Shaded}{\begin{snugshade}}{\end{snugshade}}
\newcommand{\AlertTok}[1]{\textcolor[rgb]{0.68,0.00,0.00}{#1}}
\newcommand{\AnnotationTok}[1]{\textcolor[rgb]{0.37,0.37,0.37}{#1}}
\newcommand{\AttributeTok}[1]{\textcolor[rgb]{0.40,0.45,0.13}{#1}}
\newcommand{\BaseNTok}[1]{\textcolor[rgb]{0.68,0.00,0.00}{#1}}
\newcommand{\BuiltInTok}[1]{\textcolor[rgb]{0.00,0.23,0.31}{#1}}
\newcommand{\CharTok}[1]{\textcolor[rgb]{0.13,0.47,0.30}{#1}}
\newcommand{\CommentTok}[1]{\textcolor[rgb]{0.37,0.37,0.37}{#1}}
\newcommand{\CommentVarTok}[1]{\textcolor[rgb]{0.37,0.37,0.37}{\textit{#1}}}
\newcommand{\ConstantTok}[1]{\textcolor[rgb]{0.56,0.35,0.01}{#1}}
\newcommand{\ControlFlowTok}[1]{\textcolor[rgb]{0.00,0.23,0.31}{\textbf{#1}}}
\newcommand{\DataTypeTok}[1]{\textcolor[rgb]{0.68,0.00,0.00}{#1}}
\newcommand{\DecValTok}[1]{\textcolor[rgb]{0.68,0.00,0.00}{#1}}
\newcommand{\DocumentationTok}[1]{\textcolor[rgb]{0.37,0.37,0.37}{\textit{#1}}}
\newcommand{\ErrorTok}[1]{\textcolor[rgb]{0.68,0.00,0.00}{#1}}
\newcommand{\ExtensionTok}[1]{\textcolor[rgb]{0.00,0.23,0.31}{#1}}
\newcommand{\FloatTok}[1]{\textcolor[rgb]{0.68,0.00,0.00}{#1}}
\newcommand{\FunctionTok}[1]{\textcolor[rgb]{0.28,0.35,0.67}{#1}}
\newcommand{\ImportTok}[1]{\textcolor[rgb]{0.00,0.46,0.62}{#1}}
\newcommand{\InformationTok}[1]{\textcolor[rgb]{0.37,0.37,0.37}{#1}}
\newcommand{\KeywordTok}[1]{\textcolor[rgb]{0.00,0.23,0.31}{\textbf{#1}}}
\newcommand{\NormalTok}[1]{\textcolor[rgb]{0.00,0.23,0.31}{#1}}
\newcommand{\OperatorTok}[1]{\textcolor[rgb]{0.37,0.37,0.37}{#1}}
\newcommand{\OtherTok}[1]{\textcolor[rgb]{0.00,0.23,0.31}{#1}}
\newcommand{\PreprocessorTok}[1]{\textcolor[rgb]{0.68,0.00,0.00}{#1}}
\newcommand{\RegionMarkerTok}[1]{\textcolor[rgb]{0.00,0.23,0.31}{#1}}
\newcommand{\SpecialCharTok}[1]{\textcolor[rgb]{0.37,0.37,0.37}{#1}}
\newcommand{\SpecialStringTok}[1]{\textcolor[rgb]{0.13,0.47,0.30}{#1}}
\newcommand{\StringTok}[1]{\textcolor[rgb]{0.13,0.47,0.30}{#1}}
\newcommand{\VariableTok}[1]{\textcolor[rgb]{0.07,0.07,0.07}{#1}}
\newcommand{\VerbatimStringTok}[1]{\textcolor[rgb]{0.13,0.47,0.30}{#1}}
\newcommand{\WarningTok}[1]{\textcolor[rgb]{0.37,0.37,0.37}{\textit{#1}}}

\providecommand{\tightlist}{%
  \setlength{\itemsep}{0pt}\setlength{\parskip}{0pt}}\usepackage{longtable,booktabs,array}
\usepackage{calc} % for calculating minipage widths
% Correct order of tables after \paragraph or \subparagraph
\usepackage{etoolbox}
\makeatletter
\patchcmd\longtable{\par}{\if@noskipsec\mbox{}\fi\par}{}{}
\makeatother
% Allow footnotes in longtable head/foot
\IfFileExists{footnotehyper.sty}{\usepackage{footnotehyper}}{\usepackage{footnote}}
\makesavenoteenv{longtable}
\usepackage{graphicx}
\makeatletter
\def\maxwidth{\ifdim\Gin@nat@width>\linewidth\linewidth\else\Gin@nat@width\fi}
\def\maxheight{\ifdim\Gin@nat@height>\textheight\textheight\else\Gin@nat@height\fi}
\makeatother
% Scale images if necessary, so that they will not overflow the page
% margins by default, and it is still possible to overwrite the defaults
% using explicit options in \includegraphics[width, height, ...]{}
\setkeys{Gin}{width=\maxwidth,height=\maxheight,keepaspectratio}
% Set default figure placement to htbp
\makeatletter
\def\fps@figure{htbp}
\makeatother

\KOMAoption{captions}{tableheading}
\makeatletter
\@ifpackageloaded{caption}{}{\usepackage{caption}}
\AtBeginDocument{%
\ifdefined\contentsname
  \renewcommand*\contentsname{Table of contents}
\else
  \newcommand\contentsname{Table of contents}
\fi
\ifdefined\listfigurename
  \renewcommand*\listfigurename{List of Figures}
\else
  \newcommand\listfigurename{List of Figures}
\fi
\ifdefined\listtablename
  \renewcommand*\listtablename{List of Tables}
\else
  \newcommand\listtablename{List of Tables}
\fi
\ifdefined\figurename
  \renewcommand*\figurename{Figure}
\else
  \newcommand\figurename{Figure}
\fi
\ifdefined\tablename
  \renewcommand*\tablename{Table}
\else
  \newcommand\tablename{Table}
\fi
}
\@ifpackageloaded{float}{}{\usepackage{float}}
\floatstyle{ruled}
\@ifundefined{c@chapter}{\newfloat{codelisting}{h}{lop}}{\newfloat{codelisting}{h}{lop}[chapter]}
\floatname{codelisting}{Listing}
\newcommand*\listoflistings{\listof{codelisting}{List of Listings}}
\makeatother
\makeatletter
\makeatother
\makeatletter
\@ifpackageloaded{caption}{}{\usepackage{caption}}
\@ifpackageloaded{subcaption}{}{\usepackage{subcaption}}
\makeatother

\ifLuaTeX
  \usepackage{selnolig}  % disable illegal ligatures
\fi
\usepackage{bookmark}

\IfFileExists{xurl.sty}{\usepackage{xurl}}{} % add URL line breaks if available
\urlstyle{same} % disable monospaced font for URLs
\hypersetup{
  pdftitle={Génération de rapports paramétriques avec Quarto},
  pdfauthor={Ton Nom},
  colorlinks=true,
  linkcolor={blue},
  filecolor={Maroon},
  citecolor={Blue},
  urlcolor={Blue},
  pdfcreator={LaTeX via pandoc}}


\title{Génération de rapports paramétriques avec Quarto}
\author{Ton Nom}
\date{}

\begin{document}
\maketitle

\renewcommand*\contentsname{Table of contents}
{
\hypersetup{linkcolor=}
\setcounter{tocdepth}{3}
\tableofcontents
}

\section{Quarto et Markdown}\label{quarto-et-markdown}

\subsection{Vue générale}\label{vue-guxe9nuxe9rale}

\begin{itemize}
\tightlist
\item
  Quarto repose sur \textbf{Pandoc} : convertisseur universel Markdown
  vers plusieurs formats (HTML, PDF, Word, etc.).
\item
  Écrit en Rust, moderne et rapide.
\item
  Supporte plusieurs langages dans les chunks : R, Python, Observable
  JS, etc.
\end{itemize}

\subsection{Workflow}\label{workflow}

\begin{itemize}
\tightlist
\item
  Quarto exécute les chunks de code, produit un AST (Abstract Syntax
  Tree).
\item
  Pandoc convertit cet AST en documents finaux (HTML, PDF, etc.).
\end{itemize}

\subsection{Comparaison avec R
Markdown}\label{comparaison-avec-r-markdown}

\begin{longtable}[]{@{}
  >{\raggedright\arraybackslash}p{(\columnwidth - 4\tabcolsep) * \real{0.2716}}
  >{\raggedright\arraybackslash}p{(\columnwidth - 4\tabcolsep) * \real{0.3580}}
  >{\raggedright\arraybackslash}p{(\columnwidth - 4\tabcolsep) * \real{0.3704}}@{}}
\toprule\noalign{}
\begin{minipage}[b]{\linewidth}\raggedright
Fonctionnalité
\end{minipage} & \begin{minipage}[b]{\linewidth}\raggedright
R Markdown
\end{minipage} & \begin{minipage}[b]{\linewidth}\raggedright
Quarto
\end{minipage} \\
\midrule\noalign{}
\endhead
\bottomrule\noalign{}
\endlastfoot
Langage & R + Markdown & Rust + Markdown \\
Support chunks & R uniquement & R, Python, JS, autres \\
Paramétrage YAML & Basique & Avancé \\
Sorties & HTML, PDF, Word & HTML, PDF, Word, et plus \\
\end{longtable}

\subsection{Création PDF}\label{cruxe9ation-pdf}

\begin{itemize}
\tightlist
\item
  Plusieurs méthodes :

  \begin{itemize}
  \tightlist
  \item
    LaTeX classique (fiable, mais lourd parfois)
  \item
    Typst (nouveau langage, plus rapide)
  \item
    PageJS (génération PDF via Chrome)
  \item
    Impression design via HTML/Figma
  \end{itemize}
\end{itemize}

\subsection{Exemple de chunk R}\label{exemple-de-chunk-r}

\begin{Shaded}
\begin{Highlighting}[]
\CommentTok{\#library(readxl)}
\CommentTok{\#tables \textless{}{-} read\_excel("tables\_exported.xlsx")}
\CommentTok{\#head(tables)}
\end{Highlighting}
\end{Shaded}

\section{🖨️ Quatre manières de générer un
PDF}\label{quatre-maniuxe8res-de-guxe9nuxe9rer-un-pdf}

\subsection{1. LaTeX}\label{latex}

\begin{itemize}
\tightlist
\item
  Moteur historique, très utilisé en économie, physique, et sciences
  formelles.
\item
  Complexe à utiliser, peu flexible graphiquement.
\item
  Intégration dans Quarto via \texttt{engine:\ pdf} avec
  \texttt{pdf-engine:\ lualatex}.
\item
  Permet une très grande personnalisation, mais au prix d'une courbe
  d'apprentissage raide.
\end{itemize}

\subsection{2. Typst}\label{typst}

\begin{itemize}
\tightlist
\item
  Écrit en \textbf{Rust}, pensé comme une alternative rapide et moderne
  à LaTeX.
\item
  Syntaxe simple, compilation ultra-rapide (quelques ms).
\item
  Intégré directement dans Quarto via \texttt{engine:\ typst}.
\item
  Encore jeune (2 ans), mais très prometteur.
\item
  Parfait pour des cas d'usage type rapport régulier, document
  structuré, avec un design simple.
\end{itemize}

💡 Typst est particulièrement adapté aux workflows automatisés en R ou
Python, pour produire rapidement des documents PDF paramétrés.

\subsection{3. CSS Paged Media (paged.js, weasyprint,
pdf-raptor)}\label{css-paged-media-paged.js-weasyprint-pdf-raptor}

\begin{itemize}
\tightlist
\item
  Génère d'abord un \textbf{HTML avec CSS}.
\item
  Utilise des standards CSS Media Queries + des \textbf{polyfills} pour
  structurer les pages comme des documents imprimables.
\item
  Compatible avec Chrome/Edge, et utilisable via impression PDF.
\item
  Exemples :

  \begin{itemize}
  \tightlist
  \item
    \href{https://pagedjs.org}{\texttt{paged.js}}
  \item
    \href{https://weasyprint.org}{\texttt{weasyprint}} (en Python)
  \item
    \texttt{pdf-raptor} (service web de conversion)
  \end{itemize}
\end{itemize}

\subsection{4. Génération manuelle via Figma ou HTML
pur}\label{guxe9nuxe9ration-manuelle-via-figma-ou-html-pur}

\begin{itemize}
\tightlist
\item
  Utilisé pour un \textbf{design sur-mesure} pixel-perfect.
\item
  Génère le HTML ou maquette à la main, ou via outils comme
  \textbf{Figma}.
\item
  Bypasse entièrement Pandoc ou Quarto.
\item
  Nécessite plus de maintenance, mais offre un \textbf{contrôle absolu}.
\end{itemize}

\begin{center}\rule{0.5\linewidth}{0.5pt}\end{center}

\section{Structuration d'un rapport
paramétré}\label{structuration-dun-rapport-paramuxe9truxe9}

\subsection{Paramètres définis dans l'en-tête
YAML}\label{paramuxe8tres-duxe9finis-dans-len-tuxeate-yaml}

\begin{Shaded}
\begin{Highlighting}[]
\FunctionTok{params}\KeywordTok{:}
\AttributeTok{  }\FunctionTok{country}\KeywordTok{:}\AttributeTok{ }\StringTok{"Somalia"}
\AttributeTok{  }\FunctionTok{sector}\KeywordTok{:}\AttributeTok{ }\StringTok{"Éducation"}
\AttributeTok{  }\FunctionTok{unit}\KeywordTok{:}\AttributeTok{ }\StringTok{"Région"}
\end{Highlighting}
\end{Shaded}

\subsection{Réutilisation dans le
texte}\label{ruxe9utilisation-dans-le-texte}

Les paramètres définis dans l'en-tête YAML peuvent être appelés
dynamiquement avec \texttt{params\$...}.

\begin{Shaded}
\begin{Highlighting}[]
\FunctionTok{glue}\NormalTok{(}\StringTok{"Ce rapport analyse le secteur **\{params$sector\}** pour le pays **\{params$country\}**, à l’échelle **\{params$unit\}**."}\NormalTok{)}
\end{Highlighting}
\end{Shaded}

\begin{verbatim}
Ce rapport analyse le secteur **Éducation** pour le pays **Somalia**, à l’échelle **Région**.
\end{verbatim}




\end{document}
